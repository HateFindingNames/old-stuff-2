\chapter{Execution}
\section{GMT Characteristics}
In this experiment the characteristics of the Geiger-Müller tube are investigated. For this purpose, the protective cap is first removed from the mica window. The radioactive sample is then removed from the protective cap and clamped in the holder provided on the mounting plate. The distance between the front edge of the radioactive sample and the front edge of the Geiger-Müller tube is set to $d = 5\ \mathrm{cm}$. For this purpose, a steel ruler is used as displayed in the practical room. If not already done, connect the oscilloscope to the test point and ground and switch it on. It is adjusted so that it triggers correctly. The rotary wheel of the potentiometer is turned to center position. The scaling on the oscilloscope is changed so that a peak takes up the entire height of the screen. This allows an exact reading. The oscilloscope image is photographed for documentation purposes.\\
The potentiometer is now turned to the left stop. This corresponds to $U_{GMT} = 200\ \mathrm{V}$. The LCD does not yet show any values at this voltage. The potentiometer is turned up iteratively and the measured values are recorded. The step size is adjusted accordingly. A small step size is selected for the range in which the number of pulses increases sharply. A large step size is selected on the plateau. The size of the steps used can be taken from the tab / graph \ref{} with the measured values. $U_{Start}$ is the voltage at which the first pulse can be detected on the oscilloscope.
%
\section{Angular Dependency of the Count Rate}
To investigate the influence of the alignment of the Geiger-Müller tube, it is set at a distance of $d = 5\ \mathrm{cm}$ from the radioactive source. Locking points are already provided on the mounting plate in the range -45° to +45° with 15° increments each. The radioactive sample remains straight throughout the entire experiment. The Geiger-Müller counter tube is inserted into the various positions one after the other and aligned with the radioactive source following the lines on the mounting plate. In each position 12 measurements are taken with a running time of 10 s. 
%
\section{Absorption Characteristics of Materials}
In this experiment the shielding effect of different samples is investigated. For this purpose the Geiger-Müller tube is aligned in 0° position. There is a slot on the mounting plate into which the test materials can be inserted. The materials are:
\begin{enumerate}
	\item Aluminium
	\item Lead
	\item Tin
	\item Acylic glass
	\item Cardboard
\end{enumerate}
The thickness of the materials is 2 mm each. For every material 12 measurements are taken, which are recorded in a period of 10 s each.
%
\section{Counting Statistics} \label{sec:count_stat}
For this experiment it is useful to achieve a high count rate. The Geiger-Müller tube is placed in 0° position again. This time the distance is reduced to $d = 1\ \mathrm{cm}$. To increase the counting rate of the Geiger-Müller tube the voltage is turned up. A voltage of $U_{GMT} = 500\ \mathrm{V}$ is used. 90 measurements of 10 s each are taken.
%
\section{Background Radiation}
To measure the background radiation, the experiment of \ref{sec:count_stat} is repeated. Unlike before, this time the radioactive source is removed and brought back into the protective vessel.
%
\section{Natural Radioactivity}
%
The radioactivity of Brazil nuts is investigated. For this purpose a beaker is filled with Brazil nuts and the Geiger-Müller tube is carefully inserted into the glass with the mica window facing the nuts. 90 measurements of 10 s each are recorded. The voltage remains $U_{GMT} = 500\ \mathrm{V}$.
