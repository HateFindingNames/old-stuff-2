\chapter{Execution}
\section{GMT Characteristics}
As the experiment is now being carried out, the protective cap on the mica window must be removed. This is done with great care to avoid damage. Next, the radioactive sample is clamped in the appropriate device. The sample is carefully removed from the protective vessel for this purpose. For safety reasons, avoid directing the side where the radiation is emitted towards yourself and others. For this first test, a distance of $d = 5\ \mathrm{cm}$ between the front edge of the radioactive sample and the mica window must be set. The scale on the mounting plate provides a first indication of this. However, in order to achieve greater accuracy, a steel ruler is used to measure the distance. The available oscilloscope is now connected to the test point on one side and to ground on the other side. The potentiometer is turned to the middle position. The oscilloscope shows some small peaks. The display is adjusted so that a deflection spans the whole height of the screen and triggers the oscilloscope correctly. This makes an exact reading easier. A photo of the oscilloscope image is taken. This is shown in fig. \ref. The following data can be read from it:\\
%itemize the data here or alternatively in the evaluation section
From this the maximum count rate can be determined.\\
%as well here or in the evaluation section
The potentiometer is now turned to the left stop. This corresponds to $U_{GMT} = 200\ \mathrm{V}$. The LCD does not yet show any values at this voltage. The potentiometer is turned up iteratively and the measured values are recorded. The step size is adjusted accordingly. A small step size is selected for the range in which the number of pulses increases sharply. A large step size is selected on the plateau. The size of the steps used can be taken from the tab / graph \ref with the measured values. $U_{Start}$ is the voltage at which the first pulse can be detected on the oscilloscope.
